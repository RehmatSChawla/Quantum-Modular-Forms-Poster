% Gemini theme
% https://github.com/anishathalye/gemini
%
% We try to keep this Overleaf template in sync with the canonical source on
% GitHub, but it's recommended that you obtain the template directly from
% GitHub to ensure that you are using the latest version.

\documentclass[final,10pt]{beamer}

% ====================
% Packages
% ====================
\usefonttheme{professionalfonts}
\usefonttheme{serif}
\usepackage{ragged2e}
\usepackage[T1]{fontenc}
\usepackage{lmodern}
\usepackage[orientation=portrait,size=a0,scale=1.2]{beamerposter}
\usetheme{gemini}
\usecolortheme{gemini}
\usepackage{graphicx}
\usepackage{booktabs}
\usepackage{amsmath}
\usepackage{epsfig}
\usepackage{epstopdf}


\usepackage{epsf}
\input{epsf.sty}
\usepackage{graphicx,amsmath,amssymb}
\usepackage{graphicx}
%\usepackage[export]{adjustbox}
\usepackage{epsfig,multicol}
\usepackage{multirow}
\usepackage{amsmath}

\hypersetup{
  colorlinks=true,
  citecolor=magenta,
  linkcolor=blue,
  urlcolor=violet
 }
 \usepackage{tikz}
\usetikzlibrary{arrows.meta}
\usetikzlibrary{angles,quotes} % for pic
\usetikzlibrary{decorations.markings,arrows.meta}
\tikzset{midarr/.style={decoration={markings,mark=at position #1 with {\arrow{stealth}}},postaction={decorate}},
  midarr/.default=0.5 }
\usetikzlibrary{shapes.geometric, arrows}
\usetikzlibrary{shapes,positioning}
\tikzstyle{startstop} = [rectangle, rounded corners, minimum width=2cm, minimum height=1cm,text centered, text width=4.4cm, draw=black, fill=white!30]
\tikzstyle{io} = [trapezium, trapezium left angle=70, trapezium right angle=110, minimum width=2cm, minimum height=1cm, text centered, draw=black]
\tikzstyle{process} = [rectangle, minimum width=1.5cm, minimum height=1cm, text centered,, text width=4cm, draw=black, fill=white!30]
\tikzstyle{decision} = [diamond, minimum width=3cm, minimum height=1cm, text centered, text width=4.2cm, draw=black, fill=green!30]
\tikzstyle{arrow} = [thick,->,>=stealth]

\usepackage[framemethod=TikZ]{mdframed}
\usepackage{lipsum}

\usetikzlibrary{arrows}
\usetikzlibrary{shapes}

\usepackage{stackengine}
\usepackage{pgfplots}
\stackMath
\pgfplotsset{compat=1.14}


% ====================
% Lengths
% ====================

% If you have N columns, choose \sepwidth and \colwidth such that
% (N+1)*\sepwidth + N*\colwidth = \paperwidth
\newlength{\sepwidth}
\newlength{\colwidth}
% \setlength{\sepwidth}{0.025\paperwidth}
% \setlength{\colwidth}{0.3\paperwidth}

\setlength{\sepwidth}{0.033\paperwidth}
\setlength{\colwidth}{0.45\paperwidth}


\newcommand{\separatorcolumn}{\begin{column}{\sepwidth}\end{column}}
\newcommand{\squeezeup}{\vspace{-4mm}}
% ====================
% Title
% ====================

\title{\boldmath Quantum Modular Forms in Knot Theory}

\author{Kabir Bajaj\inst{1}, Rehmat Singh Chawla\inst{1}}
\institute{\inst{1} Department of Physics, Indian Institute of Technology Bombay}


\footercontent{
  \hfill Symphy 2024}
  
% (can be left out to remove footer)

% ====================
% Logo (optional)
% ====================

% use this to include logos on the left and/or right side of the header:
% \logoright{\includegraphics[height=7cm]{logo1.pdf}}
% \logoleft{\includegraphics[height=7cm]{logo2.pdf}}
\newcommand{\mathsym}[1]{{}}
\newcommand{\unicode}[1]{{}}
\newcommand{\tr}{\operatorname{Tr}}
\newcommand{\ad}{\operatorname{ad}}
\newcommand{\vect}[1]{\boldsymbol{#1}}
\newcommand{\R}{\mathbb{R}}
\newcommand{\C}{\mathbb{C}}
\newcommand{\N}{\mathbb{N}}
\newcommand{\Z}{\mathbb{Z}}
\newcommand{\Q}{\mathbb{Q}}
\newcommand{\I}{\mathbb{I}}
\newcommand{\calL}{\mathcal{L}}
% ====================
% Body
% ====================
\begin{document}

\addtobeamertemplate{headline}{}
{
    \begin{tikzpicture}[remember picture,overlay]
      \node [anchor=north west, inner sep=3cm] at ([xshift=106.5cm,yshift=2cm]current page.north west)
      {\includegraphics[height=8.6cm]{logo.png}};
    \end{tikzpicture}
}
\begin{frame}[t]
\begin{columns}[t]
\separatorcolumn

\begin{column}{\colwidth}


\begin{block}{Motivation}
Knot Theory formalises intuitive notions of knots and links. The central problem in Knot Theory is to characterise and differentiate knots using topological quantities which are invariant for a particular knot but can differ between knots. This review explores the connection between knot invariants and curious mathematical objects called Quantum Modular Forms.
\end{block}


\begin{block}{Modular Forms}
\begin{itemize}
    \item The Modular Group is $\text{PSL}_2(\Z)=\text{SL}_2(\Z)/\{\pm \I\}$ which acts as $\left(\!\begin{smallmatrix}a&b\\c&d\end{smallmatrix}\!\right) z = \frac{az+b}{cz+d}$ on $z\in\C\cup\{\infty\}$.
    \item 
    \item A modular form is a holomorphic function on $H$(upper half complex plane)$\cup\{\infty\}$ transforming under the modular group as $f(z)=(cz+d)^{-2k}f(gz)\; \forall g\in\mathrm{SL}_2(\Z)$.
\end{itemize}
\end{block}


\begin{block}{Quantum Modular Forms}
\end{block}



\begin{block}{Knot Theory}
\begin{itemize}
    \item A knot is a closed path embedded in a 3-manifold, generally $\R^3$.
    \item A link is a collection of knots, possibly intertwined. Knot invariants can also be generalised to link invariants.
    \item A famous knot invariant is the Jones Polynomial, a Laurent polynomial generated via a recursive relation between knots related by the addition of a single crossing.
    \item Witten showed that the Jones polynomial for a knot can be obtained as the expectation value of the corresponding Wilson loop operator in a Chern-Simons field theory with gauge group $SU(2)$.
    \item The coloured Jones polynomials are generalisations obtained from the $SU(N)$ Chern-Simons theories.
\end{itemize}
[] Add examples of knots\\
[] Explain Torus knots
\end{block}

\end{column}
\separatorcolumn


\begin{column}{\colwidth}
\begin{block}{Hikami and Lovejoy, 2014}
    
\end{block}

\end{column}

\separatorcolumn


\end{columns}
\end{frame}


\end{document}
